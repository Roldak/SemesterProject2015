\section{The Miniboxing Transformation}
\label{sec:minibox}

% Summary
This section describes the miniboxing transformation, showing why it is necessary, how it improves the program performance and how it has an opportunistic nature.

\subsection{Generics in Scala}

% Generics => Different data types & sizes => Erasure => Example
Generics are crucial to productivity. They allow programmers to design algorithms and data structures that operate identically regardless of the data used, fostering code reuse. For example, a |Vector[T]| can be instantiated for any type, whether for numbers, strings or CPU threads. However, on the low level, data comes in different shapes and sizes: from 1-bit booleans to 64-bit long integers, floating point numbers, characters, value classes \cite{gosling-value-classes,rose-value-classes-tearing,sip-value-classes} and objects. In a |Vector[Int]|, getters and setters receive integer values of 32-bits while in |Vector[Double]| they receive 64-bit double-precision floating-point numbers.

The current compilation scheme for generics is called erasure, and is the simplest compilation scheme possible for generics. Erasure requires all data, regardless of its type, to be passed in by reference, pointing to a heap object. Let us take a simple example, a generic |identity| method written in Scala:

\begin{lstlisting-nobreak}
 def identity[T](t: T): T = t
 val five = identity(5)
\end{lstlisting-nobreak}

When compiled, the source code corresponding to the low-level bytecode for the method is:

\begin{lstlisting-nobreak}
 def identity(t: `Object`) = `Object`
\end{lstlisting-nobreak}

As the name suggests, the type parameter |T| was ``erased'' from the method, leaving it to accept and return |Object|s. The problem with this approach is that values of primitive types, such as integers, need to be transformed into heap objects when passed to generic code, so they are compatible with |Object|, in a process called boxing. The process goes two-ways: the return of generic methods needs to be unboxed back to a primitive type:

\begin{lstlisting-nobreak}
 val five = identity(`Integer.valueOf(5)`).`intValue()`
\end{lstlisting-nobreak}

% Naive approach => specialization => Example
Boxing primitive types requires heap allocation and garbage collection, both of which slow down program performance. Furthermore, when values are stored in generic classes, such as |Vector[T]|, they need to be stored in the boxed format, thus inflating the heap memory requirements and slowing down execution. In practice, generic methods can be as much as 10 times slower than their monomorphic (primitive) instantiations. This gave rise to a simple and effective idea: specialization.

\subsection{Specialization}

Specialization is a second approach used by the Scala compiler to translate generics. It is triggered by the |@specialized| annotation added to a type parameter:

\begin{lstlisting-nobreak}
 def identity[`@specialized` T](t: T): T = t
 val five = identity(5)
\end{lstlisting-nobreak}

Based on the annotation, the specialization transformation creates several versions of the |identity| method:

\begin{lstlisting-nobreak}
 def identity(t: Object): Object = t
 def identity`_I`(t: int): int = t
 def identity`_C`(t: char): char = t
 // ... and another 7 versions of the method
\end{lstlisting-nobreak}

Having multiple methods, also called specialized variants or simply specializations of the |identity| method, the compiler can optimize the call to |identity|:

\begin{lstlisting-nobreak}
 val five: int = identity`_I`(5)
\end{lstlisting-nobreak}

% Used in Scala => limited by Duplication
This transformation side-steps the need for a heap object allocation, improving the program performance.
However, specialization is not without limitations. As we have seen, it creates 10 versions of the method for each type parameter. But what if a method has 2 type parameters? It creates 100 versions, and in general, for $N$ specialized type parameters, it creates $10^N$ specialized variants, the cartesian product covering all combinations. This prevents the Scala library from using specialization extensively, since common classes have between one and three type parameters. This led to the next development, the miniboxing transformation.

\subsection{Miniboxing}

% Miniboxing encoding => Example
Taking a low level perspective, we can observe the fact that all primitive types in the Scala programming language fit within 64 bits. This is the main idea that motivated the miniboxing transformation: instead of creating separate versions of the code for each primitive type alone, we can create a single one, which stores 64-bit encoded values, much like a tagged union \cite{tagged-unions-lua}. But unlike a tagged union, miniboxing can use the statically typed nature of the language to avoid carrying tags with each value. Looking at the previous example:

\begin{lstlisting-nobreak}
 def identity[`@miniboxed` T](t: T): T = t
 val five = identity(5)
\end{lstlisting-nobreak}

Since the type parameter is annotated with |@miniboxed|, the compiler\footnote{This occurs in the presence of the miniboxing plugin attached to the Scala compiler. In the rest of the paper we assume the miniboxing Scala complier plugin is active unless otherwise noted. For more information on adding the miniboxing plugin to the build please see \url{http://scala-miniboxing.org}.} translates the code to:

\begin{lstlisting-nobreak}
 def identity(t: Object): Object = t
 def identity`_M`(..., t: long): long = t
 val five: int= `minibox2int`(identity`_M`(`int2minibox`(5)))
\end{lstlisting-nobreak}

% Cost of miniboxing
Alert readers will notice the |minibox2int| and |int2minibox| transformations act exactly like the boxing coercions in the case of  erased generics. This is true, the values are being coerced to the miniboxed representation, much like in the case of erasure, but on the Java Virtual Machine platform our benchmarks have shown that the miniboxing conversion cost is completely eliminated when compiling the code to native x86 assembly. Further benchmarking has shown that the code matches the performance of specialized code within a 10\% slowdown due to coercions \cite{miniboxing}.

% Ellipsis => Type tags + send to the papers
There is an elipsis in the definition of the |identity_M| method, which stands for what we call a type byte: a byte describing the type encoded in the long integer, allowing the operations such as |toString|, |hashCode| or |equals| to be executed correctly on encoded values, treating them as the original primitive (corresponding to |T|) rather than long integers. Although the transformation looks simple, there are many subtleties we have to take into account when transforming the code \cite{miniboxing,miniboxing-linkedlist,ldl}.

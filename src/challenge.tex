
\section{Challenging the Assumptions}

In this section, I describe the steps that I followed in order to debug the slowdown observed in the last section, that is, to identify its cause.

\subsection{Interpretation of the Numbers}

First of all, I tried to tweak the parameters of the benchmark to try deducing the cause of the slowdowns. As one could expect, modifying basic items such as the size of the |ArrayBuffer|s did not improve the numbers. Ultimately, I had to dig deeper into what could cause the problem, and to this end rewrote the benchmarking process by hand so that it would be easier to debug and to experiment with the JVM flags. Thinking back about the original problem, one of the cause for the slowdown could have been that hot methods would not get inlined properly for some reasons. Unfortunately, running the benchmark with the |-XX:+PrintCompilation| JVM flag did not confirm the hypothesis. On the other hand, running the benchmark once more with the |-XX:+PrintGC| option yielded a more interesting result: the version using |MbArray|s would spend way more time collecting garbage than the version using |ClassTag|s:

\begin{figure}
\subfigure[ClassTag version]{
	\lstinputlisting{BenchmarkOutputs/CtVector1.txt}
	\label{fig:CtVector1}
}
\subfigure[Miniboxed version]{
	\lstinputlisting{BenchmarkOutputs/MbVector1.txt}
    \label{fig:MbVector1}
}
\caption{Benchmark outputs for ArrayBuffers of 10'000'000 elements}
\label{fig:GcComp}
\end{figure}

In figure \ref{fig:GcComp}, we can see that during one iteration of the benchmark:
\begin{enumerate}
  \item The |ClassTag| version spends only |~66ms| collecting garbage.
  \item The |MbArray| version spends up to |~145ms| collecting garbage. 
\end{enumerate} 

\subsection{Problem with the Current Design}

In order to understand the cause of the slowdown, it is necessary to get a hang of how |MbArray|s are currently transformed by the miniboxing plugin.

When an |MbArray| is instantiated in a context where its type parameter is known to be a primitive type, which can only happen when the instantiation is monomorphic or occurs in a miniboxed generic context, the miniboxing plugin will at compile time transform the instantiation of the generic |MbArray| into one of its specialized version: |MbArray_J| for an integral type, or |MbArray_D| for a floating point type.
At first, it seemed like a good idea to have only these two specialized versions: since the miniboxing plugin only deals with two specialized accessors anyway -- |MbArray_apply_J| and |MbArray_apply_D| -- we cannot gain any performance (in term of number of instruction executed) by adding other versions, since their values would have to get coerced into their respective minibox, |Long| or |Int|, in order get returned by |MbArray_apply_J| or |MbArray_apply_D|.  
Well, it turns out that this is actually a problem, and having only two variants \emph{does} affect performance. Except, it is not about number of instructions executed, but the memory footprint left by the program. Indeed, having only |MbArray_J| to store values of integral type runs great while the actual data type is |Long|. However, it will use 2 times the memory needed to store |Int|s, and will use up to 64 times the memory needed to store |Boolean|s. Not only is this not acceptable in terms of memory consumption, but this also affects runtime performance by triggering more GC cycles and thus increasing the total time spent in collecting garbage as it has been shown above.

%Now what remains to be done before being able to fix the issue is to explain how the difference in GC time could be so important:
%The first things that stands out is how memory consumption for the |MbArray| version gets up to $800MB$, whereas the |ClassTag| version only reaches $600MB$.

